\documentclass{article}
\usepackage[utf8]{inputenc}

\title{Responses for the Reading Questions 3: Reduced Row-Echelon Form}
\author{Yong Hoon, Do}
\date{\today}

\usepackage{amsmath}

\begin{document}

\maketitle


%--------------------------------------------------------------------------------

\section{Is the matrix below in reduced row-echelon form? Why or why not?}

\[ \left| \begin{array}{cccccc}
1 & 5 & 0 & 6 & 8 \\
0 & 0 & 1 & 2 & 0  \\
0 & 0 & 0 & 0 & 1
\end{array} \right|\]

The given matrix is reduced row-echelon form because it meets every following conditions:

\begin{itemize}
    \item the leftmost nonzero entry row at each row is equal to $1$~,
    \item and also the leftmost columns that have nonzero entry are the pivot columns.
\end{itemize}

Thus,
\[r=3\]
\[D=\{1,3,5\}\]
\[F=\{2,4\}\]

%--------------------------------------------------------------------------------

\section{Use row operations to convert the matrix below to reduced row-echelon form and report the final matrix.}

The given matrix is the $3 \times 3$~matrix

\[
\left| \begin{array}{ccc}
2 & 1 & 8 \\
-1 & 1 & -1  \\
-2 & 5 & 4
\end{array} \right|
\]

\bigskip

We can transform the matrix according to the definition of Row Operations

\[
\left| \begin{array}{ccc}
2 & 1 & 8 \\
-1 & 1 & -1  \\
-2 & 5 & 4
\end{array} \right|
\xrightarrow[]{\text{$R\textsubscript{1} \leftrightarrow R\textsubscript{2}$~}}
\left| \begin{array}{ccc}
-1 & 1 & -1  \\
2 & 1 & 8 \\
-2 & 5 & 4
\end{array} \right|
\xrightarrow[]{\text{$ (2 \times R\textsubscript{1}) + R\textsubscript{2} $~}}
\left| \begin{array}{ccc}
-1 & 1 & -1  \\
0 & 3 & 6 \\
-2 & 5 & 4
\end{array} \right|
\]

\[
\xrightarrow[]{\text{$ (-2 \times R\textsubscript{1}) + R\textsubscript{3} $~}}
\left| \begin{array}{ccc}
-1 & 1 & -1  \\
0 & 3 & 6 \\
0 & 3 & 6
\end{array} \right|
\xrightarrow[]{\text{$ (-R\textsubscript{2}) + R\textsubscript{3} $~}}
\left| \begin{array}{ccc}
-1 & 1 & -1  \\
0 & 3 & 6 \\
0 & 0 & 0
\end{array} \right|
\]

\[
\xrightarrow[]{\text{$ R\textsubscript{2} \div 3 $~}}
\left| \begin{array}{ccc}
-1 & 1 & -1  \\
0 & $\boxed{1}$~ & 2 \\
0 & 0 & 0
\end{array} \right|
\xrightarrow[]{\text{$ (-1 \times R\textsubscript{2}) + R\textsubscript{1} $~}}
\left| \begin{array}{ccc}
-1 & 0 & -3  \\
0 & $\boxed{1}$~ & 2 \\
0 & 0 & 0
\end{array} \right|
\]

\[
\xrightarrow[]{\text{$ (-1 \times R\textsubscript{1}) $~}}
\left| \begin{array}{ccc}
$\boxed{1}$~ & 0 & 3  \\
0 & $\boxed{1}$~ & 2 \\
0 & 0 & 0
\end{array} \right|
\]

\bigskip

Thus, the Reduced Row-Echelon Form will be

\[\left| \begin{array}{ccc}
$\boxed{1}$~ & 0 & 3  \\
0 & $\boxed{1}$~ & 2 \\
0 & 0 & 0
\end{array} \right|\]

where
\[r=2\]
\[D=\{1,2\}\]
\[F=\{3\}\]



%--------------------------------------------------------------------------------

\section{Find all the solutions to the system below by using an augmented matrix and row operations. Report your final matrix in reduced row-echelon form and the set of solutions.}

Let us find a set of the solutions to the following system of equations,

\[
\begin{cases}
2x\textsubscript{1} + 3x\textsubscript{2} - x\textsubscript{3} = 0 \\
x\textsubscript{1} + 2x\textsubscript{2} + x\textsubscript{3} = 3 \\
x\textsubscript{1} + 3x\textsubscript{2} + 3x\textsubscript{3} = 7
\end{cases}
\]

\bigskip

First, form the augmented matrix,

\[
\left[
\begin{array}{cccc}
2 & 3 & -1 & 0 \\
1 & 2 & 1 & 3 \\
1 & 3 & 3 & 7 \\
\end{array}
\right]
\]

\bigskip

and work to reduced row-echelon form,

\[
\xrightarrow[]{\text{$ R\textsubscript{1} \leftrightarrow R\textsubscript{2} $~}}
\left| \begin{array}{cccc}
$\boxed{1}$~ & 2 & 1 & 3 \\
2 & 3 & -1 & 0 \\
1 & 3 & 3 & 7 \\
\end{array} \right|
\xrightarrow[]{\text{$ (-2 \times R\textsubscript{1}) + R\textsubscript{2} $~}}
\left| \begin{array}{cccc}
$\boxed{1}$~ & 2 & 1 & 3 \\
0 & -1 & -3 & -6 \\
1 & 3 & 3 & 7 \\
\end{array} \right|
\]

\[
\xrightarrow[]{\text{$ (-1 \times R\textsubscript{1}) + R\textsubscript{3} $~}}
\left| \begin{array}{cccc}
$\boxed{1}$~ & 2 & 1 & 3 \\
0 & -1 & -3 & -6 \\
0 & 1 & 2 & 4 \\
\end{array} \right|
\xrightarrow[]{\text{$ R\textsubscript{2} + R\textsubscript{3} $~}}
\left| \begin{array}{cccc}
$\boxed{1}$~ & 2 & 1 & 3 \\
0 & -1 & -3 & -6 \\
0 & 0 & -1 & -2 \\
\end{array} \right|
\]

\[
\xrightarrow[]{\text{$ -1 \times R\textsubscript{3} $~}}
\left| \begin{array}{cccc}
$\boxed{1}$~ & 2 & 1 & 3 \\
0 & -1 & -3 & -6 \\
0 & 0 & $\boxed{1}$~ & 2 \\
\end{array} \right|
\xrightarrow[]{\text{$ (3 \times R\textsubscript{3}) + R\textsubscript{2} $~}}
\left| \begin{array}{cccc}
$\boxed{1}$~ & 2 & 1 & 3 \\
0 & -1 & 0 & 0 \\
0 & 0 & $\boxed{1}$~ & 2 \\
\end{array} \right|
\]

\[
\xrightarrow[]{\text{$ (-1 \times R\textsubscript{3}) + R\textsubscript{1} $~}}
\left| \begin{array}{cccc}
$\boxed{1}$~ & 2 & 0 & 1 \\
0 & -1 & 0 & 0 \\
0 & 0 & $\boxed{1}$~ & 2 \\
\end{array} \right|
\xrightarrow[]{\text{$ -1 \times R\textsubscript{2} $~}}
\left| \begin{array}{cccc}
$\boxed{1}$~ & 2 & 0 & 1 \\
0 & $\boxed{1}$~ & 0 & 0 \\
0 & 0 & $\boxed{1}$~ & 2 \\
\end{array} \right|
\]

\[
\xrightarrow[]{\text{$ (-1 \times R\textsubscript{2}) + R\textsubscript{1} $~}}
\left| \begin{array}{cccc}
$\boxed{1}$~ & 0 & 0 & 1 \\
0 & $\boxed{1}$~ & 0 & 0 \\
0 & 0 & $\boxed{1}$~ & 2 \\
\end{array} \right|
\]

\bigskip

Now, the only solution to the given augmented matrix is namely, \(x\textsubscript{1}=1\), \(x\textsubscript{2}=0\), \(x\textsubscript{3}=2\). Based on this fact, we have determined the entire solution set,

\[S=
\left\{
\left| \begin{array}{c}
1 \\
0 \\
2 \\
\end{array} \right|
\right\}
\]

%--------------------------------------------------------------------------------

\end{document}
