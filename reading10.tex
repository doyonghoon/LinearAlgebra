\documentclass{article}
\usepackage[utf8]{inputenc}

\title{Responses to the Reading Questions 10}
\author{Yong Hoon, Do}
\date{\today}

\usepackage{bm}
\usepackage{amsmath}
\usepackage{amssymb}
\usepackage{systeme}
\usepackage{chngcntr}

\counterwithin*{equation}{section}
\counterwithin*{equation}{subsection}



\begin{document}

\maketitle

\begin{center}
Spanning Sets
\end{center}

\newcommand{\sol} {
  \textbf{Solution:}
}

\newcommand{\LIVHS} {\textbf{Linearly Independent Vectors and Homogeneous Systems}}

\newcommand{\FVCS} {\textbf{Free Variables for Consistent Systems}}

\newcommand{\HSC} {\textbf{Homogeneous Systems are Consistent}}

\newcommand{\ls} {
\(\mathcal{LS}(A,\textbf{0})\)
}

%--------------------------------------------------------------------------------
\section{Let \(S\) be the set of three vectors below.}
\newcommand{\exone}{
  S=
  \left\{
  \left|
  \begin{array}{c}
    1 \\
    2 \\
    -1 \\
  \end{array}
  \right|
  ,
  \left|
  \begin{array}{c}
    3 \\
    -4 \\
    2 \\
  \end{array}
  \right|
  ,
  \left|
  \begin{array}{c}
    4 \\
    -2 \\
    1 \\
  \end{array}
  \right|
  \right\}
}

\newcommand{\extwo}{
  S=
  \left\{
  \left|
  \begin{array}{c}
    1 \\
    -1 \\
    0 \\
  \end{array}
  \right|
  ,
  \left|
  \begin{array}{c}
    3 \\
    2 \\
    2 \\
  \end{array}
  \right|
  ,
  \left|
  \begin{array}{c}
    4 \\
    4 \\
    -4 \\
  \end{array}
  \right|
  \right\}
}

\[
\exone
\]

Is \(S\) linearly independent or linearly dependent? Explain why.

\bigskip

\sol

To determine linear independence we first form a relation of linear dependence,
\begin{equation}
  A =
  \left|
  \begin{array}{cccc}
    1 & 3 & 4 \\
    2 & -4 & -2 \\
    -1 & 2 & 1 \\
  \end{array}
  \right|
  \xrightarrow[]{\text{RREF}}
  \left|
  \begin{array}{cccc}
    $\boxed{1}$~ & 0 & 1 \\
    0 & $\boxed{1}$~ & 1 \\
    0 & 0 & 0 \\
  \end{array}
  \right|
\end{equation}

Now, \(r=2\), so there are \(n-r=3-2=1 > 0\) free variables and we see that \(\mathcal{LS}(A,\textbf{0})\) has infinitely many solutions (\LIVHS, \FVCS, \HSC), the set \(S\) is linearly dependent.

%--------------------------------------------------------------------------------
\section{Let \(S\) be the set of three vectors below.}

\[
\extwo
\]

\begin{equation}
  A =
  \left|
  \begin{array}{cccc}
    1 & 3 & 4 \\
    -1 & 2 & 4 \\
    0 & 2 & -4 \\
  \end{array}
  \right|
  \xrightarrow[]{\text{RREF}}
  \left|
  \begin{array}{cccc}
    $\boxed{1}$~ & 0 & 1 \\
    0 & $\boxed{1}$~ & 1 \\
    0 & 0 & $\boxed{1}$~ \\
  \end{array}
  \right|
\end{equation}

Now, \(r=3\), so there are \(n-r=3-3=0 > 0\) free variables and
we see that \(\mathcal{LS}(A,\textbf{0})\) has a unique solution
(\FVCS, \HSC). By the theorem of \LIVHS, the set \(S\) is linearly independent.

%--------------------------------------------------------------------------------
\section{Is the matrix below singular or nonsingular? Explain your answer using only the final conclusion you reached in the previous question, along with one new theorem.}

\[
\left|
\begin{array}{cccc}
  1 & 3 & 4 \\
  -1 & 2 & 3 \\
  0 & 2 & -4 \\
\end{array}
\right|
\]

\sol

The given matrix a nonsingular matrix since the homogeneous system, \ls, has only the trivial solution.
According to the definition of nonsingular matrices, the homogeneous system \ls has a unique solution.
So by the theorem \LIVHS, the columns of \(B\) form a linearly independent set.

\end{document}
