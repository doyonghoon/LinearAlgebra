\documentclass{article}
\usepackage[utf8]{inputenc}

\title{Responses to the Reading Questions: Linear Dependence and Spans}

\usepackage{bm}
\usepackage{amsmath}
\usepackage{amssymb}
\usepackage{systeme}
\usepackage{chngcntr}

\counterwithin*{equation}{section}
\counterwithin*{equation}{subsection}

\begin{document}

\maketitle

\newcommand{\sol} {
  \textbf{Solution:}
}

\newcommand{\LIVHS} {\textbf{Linearly Independent Vectors and Homogeneous Systems}}

\newcommand{\FVCS} {\textbf{Free Variables for Consistent Systems}}

\newcommand{\HSC} {\textbf{Homogeneous Systems are Consistent}}

\newcommand{\ls} {\(\mathcal{LS}(A,\textbf{0})\)}

\newcommand{\p} {$\boxed{1}$~}

%--------------------------------------------------------------------------------
\noindent\textbf{Question1} Let \(S\) be the linearly dependent set of three vectors below.
\[
S=
\begin{Bmatrix}
  \begin{bmatrix}
    1 \\
    10 \\
    100 \\
    1000 \\
  \end{bmatrix}
,
\begin{bmatrix}
  1 \\
  1 \\
  1 \\
  1 \\
\end{bmatrix}
,
\begin{bmatrix}
  5 \\
  23 \\
  203 \\
  2003 \\
\end{bmatrix}
\end{Bmatrix}
\]

\bigskip

\sol

The given Set \(S\) of vectors implies that  there exist scalars \(a_1, a_2, a_3\)
such that \(a_1 v_1 + a_2 v_2 + a_3 v_3 = 0\) where at least one of the \(a \neq 0\)
Writing the equation out in matrix form

\[
\begin{bmatrix}
  1 & 1 & 5 \\
  10 & 1 & 23 \\
  100 & 1 & 203 \\
  1000 & 1 & 2003
\end{bmatrix}
\begin{bmatrix}
  a_1 \\
  a_2 \\
  a_3 \\
  a_4
\end{bmatrix}
=
\begin{bmatrix}
  0 \\
  0 \\
  0 \\
  0
\end{bmatrix}
\]

and row reducing this matrix then we get
\begin{equation}
  \begin{bmatrix}
    \p & 0 & 2 \\
    0 & \p & 3 \\
    0 & 0 & 0 \\
    0 & 0 & 0 \\
  \end{bmatrix}
\end{equation}

\begin{equation}
  \begin{matrix}
    x_1=-2x_3 \\
    x_2 = -3x_3 \\
    x_3 = x_3 \\
  \end{matrix}
\end{equation}

Now, if we choose \(x_3 = 1\) we find the solution
\(
\begin{bmatrix}
  -2 \\
  -3 \\
  1
\end{bmatrix}
\)
So, we can write the relation as \(-2v_1-3v_2+v_3=0\). Solving for \(v_3\), \(v_3=3v_2+2v_1\)
Which will allow us to write \(V=<S> = <\{v_1, v_2, v_3\}> = <\{v_1, v_2\}>\)
Let \(S'=\{v_1, v_2\}\) and \(V'=<S'>\) then \(V=V'\)

1) \(V' \subseteq V\) Entry vector of \(S'\) is in \(S\)
2) \(V' \subseteq V\) There are scalars \(a_1, a_2, a_3\) such that

\begin{equation}
  \begin{matrix}
    V = a_1 v_1 + a_2 v_2 + a_3 v_3 \\
      = a_1 v_1 + a_2 v_2 + a_3 (3v_3 + 2v_1) \\
      = a_1 v_1 + a_2 v_2 + 3 a_3 v_3 + 2 a_3 v_1) \\
      = (a_1 + 2 a_3) v_1 + (a_2 + 3 a_3) v_2
  \end{matrix}
\end{equation}

\(V\) can be written as a linear combination of the vectors in \(S'\)

\bigskip
\bigskip

%--------------------------------------------------------------------------------
\noindent\textbf{Question2} Explain why the word “dependent” is used in the definition of linear dependence.

\bigskip

\sol

In any linearly dependent set there is always one vector that can be written as a linear combination of the others.
This is the substance of the upcoming Theorem \textbf{Dependency in Linearly Dependent Sets}.
Perhaps this will explain the use of the word “dependent.”

\bigskip
\bigskip

%--------------------------------------------------------------------------------
\noindent\textbf{Question3} Suppose that \(Y=<P>=<Q>\), where \(P\) is a linearly dependent set and \(Q\) is linearly dependent.
Would you rather use \(P\) or \(Q\) to describe \(Y\)? Why?

\bigskip

\sol

I would rather use the set \(Q\) because the linear independence of \(Q\) guarantees uniqueness in the solution of \(Y\)

\end{document}
