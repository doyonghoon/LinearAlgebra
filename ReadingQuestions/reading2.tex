\documentclass{article}
\usepackage[utf8]{inputenc}

\title{Responses for the Reading Questions 2: Solving Systems of Linear Equations}
\author{Yong Hoon, Do}
\date{\today}

\usepackage{amsmath}
\usepackage{amssymb}
\usepackage{systeme}

\begin{document}

\maketitle


%--------------------------------------------------------------------------------

\section{How many solutions does the system of equations $3x+2y=4$~, $6x+4y=8$~ have? Explain your answer.}

A number of the solutions to the given system of the equations is infinitely many because these two given equations are essentially same.

\bigskip

The given equations are
\[
\systeme{
  3x+2y=4,
  6x+4y=8
}
\]

\bigskip

$3x+2y=4$~ is still same with the second equation when we multiply $2$~ for the both side of the first equation, and it gives $6x+4y=8$~, which is exactly same with the second equation. Since any number for $x\in\mathbb{C}$ and $y\in\mathbb{C}$ will give the exact same values from the both equations, there are infinitely many solutions to the given system of equations. It can be also geometrically proven by sketching the graph based on the given equations because these two equations will be drawn at the same points for every $x$ and $y$.



%--------------------------------------------------------------------------------

\section{How many solutions does the system of equations $3x+2y=4$~, $6x+4y=-2$~ have? Explain your answer.}

A number of the solutions to the given system of the equations is zero because the given two equations won’t have any points $(x,y)$ in common, and we can say that set is empty, $S = \emptyset$.

%--------------------------------------------------------------------------------

\section{What do we mean when we say mathematics is a language?}

The most powerful component in mathematics as a view of a language is that we can clearly think about complicated ideas by formulating questions or answers clearly.


%--------------------------------------------------------------------------------

\end{document}
