\documentclass{article}
\usepackage[utf8]{inputenc}

\title{Corrections for Quiz 2}

\usepackage{bm}
\usepackage{amsmath}
\usepackage{amssymb}
\usepackage{systeme}
\usepackage{chngcntr}

\counterwithin*{equation}{section}
\counterwithin*{equation}{subsection}

\begin{document}

\maketitle

\newcommand{\sol} {
  \textbf{Solution:}
}

\newcommand{\LIVHS} {\textbf{Linearly Independent Vectors and Homogeneous Systems}}

\newcommand{\FVCS} {\textbf{Free Variables for Consistent Systems}}

\newcommand{\HSC} {\textbf{Homogeneous Systems are Consistent}}

\newcommand{\ls} {\(\mathcal{LS}(A,\textbf{0})\)}

\newcommand{\nullspace} {\(\mathcal{N}(A)\)}

\newcommand{\p} {$\boxed{1}$~}

%--------------------------------------------------------------------------------
\section{Question1}

According to the question, the given matrix is not in homogeneous system, but it is the matrix that we are only interested in determining whether it is singular or nonsingular.


\section{Question2}
I had a mistake on copying numbers from the system of equations when building an augmented matrix.
As a result, I had a wrong vector form.

The augmented matrix and the row reduced matrix should look like

\begin{equation}
  \begin{bmatrix}
    2 & -4 & 3 & 0 & 1 & 6 \\
    1 & -2 & -2 & 14 & -4 & 15 \\
    1 & -2 & 1 & 2 & 1 & -1 \\
    -2 & 4 & 0 & -12 & 1 & -7
  \end{bmatrix}
\xrightarrow[]{\text{RREF}}
\begin{bmatrix}
  \p & -2 & 0 & 6 & 0 & 1 \\
  0 & 0 & \p & -4 & 0 & 3 \\
  0 & 0 & 0 & 0 & \p & -5 \\
  0 & 0 & 0 & 0 & 0 & 0 \\
\end{bmatrix}
\end{equation}

Now vector form will be

\begin{equation}
  \left\{
    x_2
    \begin{bmatrix}
      2 \\
      1 \\
      0 \\
      0 \\
      0 \\
    \end{bmatrix}
    +
    x_4
    \begin{bmatrix}
      -6 \\
      0 \\
      4 \\
      1 \\
      0 \\
    \end{bmatrix}
    +
    \begin{bmatrix}
      1 \\
      0 \\
      3 \\
      0 \\
      -5 \\
    \end{bmatrix}
    |
    x_2, x_4\in\mathbb{C}
  \right\}
\end{equation}

\section{Question3}
\noindent\textbf{a)} I had a wrong solution set to the \nullspace. The correct solution set will be
\[
S=
\left\{
  \begin{vmatrix}
    x_1 = x_3 + x_4 \\
    x_2 = -x_3 - 2x_4 \\
    x_3 = x_3 \\
    x_4 = x_4
  \end{vmatrix}
  |
  x_3, x_4\in\mathbb{C}
\right\}
\]

And the correct vector form is

\[
S=
\left\{
  x_3
  \begin{vmatrix}
    1 \\
    -1 \\
    1 \\
    0
  \end{vmatrix}
  +
  x_4
  \begin{vmatrix}
    1 \\
    -2 \\
    0 \\
    1
  \end{vmatrix}
  |
  x_3, x_4\in\mathbb{C}
\right\}
\]

Thus,
\[
\mathcal{N}(A) = <S>=
\left\{
  \begin{vmatrix}
    1 \\
    -1 \\
    1 \\
    0
  \end{vmatrix}
  ,
  \begin{vmatrix}
    1 \\
    -2 \\
    0 \\
    1
  \end{vmatrix}
\right\}
\]

\noindent\textbf{c)} I had a wrong calculation based on the wrong solution set, and I picked wrong numbers for \(x_3, x_4\).
The correct answer should be

\[
  x_3
  \begin{vmatrix}
    1 \\
    -1 \\
    1 \\
    0
  \end{vmatrix}
  +
  x_4
  \begin{vmatrix}
    1 \\
    -2 \\
    0 \\
    1
  \end{vmatrix}
  =z=
  \begin{vmatrix}
    3 \\
    -5 \\
    1 \\
    2
  \end{vmatrix}
\]

We can \(z\) vector by choosing \(x_3=1, x_4=2\)
\[
z=
  \begin{vmatrix}
    1 \\
    -1 \\
    1 \\
    0
  \end{vmatrix}
  +
  \begin{vmatrix}
    2 \\
    -4 \\
    0 \\
    2
  \end{vmatrix}
=
  \begin{vmatrix}
    3 \\
    -5 \\
    1 \\
    2
  \end{vmatrix}
\]

\section{Question4}

While checking a relation of linear dependence, I skipped the step that shows how the relation of linear dependence on the given set is strongly associated with a system of equations.
Since we have to show every scalars must be \(0\) to justify the given set is a linearly independent, I had to add a step that the scalars used in a relation of linear dependence is \(0\).

\bigskip

Check a relation of linear dependence
\begin{equation}
  \alpha_1(v_1) + \alpha_2(v_1 + v_2) + \alpha_3(v_1+v_2+v_3) + \alpha_4(v_1+v_2+v_3+v_4) = 0
\end{equation}

\begin{equation}
v_1(\alpha_1) + v_2(\alpha_1 + \alpha_2) + v_3(\alpha_1 + \alpha_2 + \alpha_3) + v_4(\alpha_1 + \alpha_2 + \alpha_3 + \alpha_4) = 0
\end{equation}

Now, I can build a system of equations based on the fact that the relation of linear dependence would be equal to zero.

\begin{equation}
  \begin{matrix}
    \alpha_1 + \alpha_2 + \alpha_3 + \alpha_4 = 0 \\
    \alpha_1 + \alpha_2 + \alpha_3 = 0 \\
    \alpha_1 + \alpha_2 = 0 \\
    \alpha_1 = 0
  \end{matrix}
\end{equation}

Since \(\alpha_1 = 0\), then every scalars must be zero. Thus, the given set is a linearly independent.
\end{document}
