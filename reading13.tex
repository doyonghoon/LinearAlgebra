\documentclass{article}
\usepackage[utf8]{inputenc}

\title{Responses to the Reading Questions: Matrix Operations}

\usepackage{bm}
\usepackage{amsmath}
\usepackage{amssymb}
\usepackage{systeme}
\usepackage{chngcntr}

\counterwithin*{equation}{section}
\counterwithin*{equation}{subsection}

\begin{document}

\maketitle

\newcommand{\sol} {
  \textbf{Solution:}
}

\newcommand{\LIVHS} {\textbf{Linearly Independent Vectors and Homogeneous Systems}}

\newcommand{\FVCS} {\textbf{Free Variables for Consistent Systems}}

\newcommand{\HSC} {\textbf{Homogeneous Systems are Consistent}}

\newcommand{\ls} {\(\mathcal{LS}(A,\textbf{0})\)}

\newcommand{\p} {$\boxed{1}$~}

\newcommand{\QAA} {
\begin{bmatrix}
  2 & -2 & 8 & 1 \\
  4 & 5 & -1 & 3 \\
  7 & -3 & 0 & 2 \\
\end{bmatrix}
}

\newcommand{\QAB} {
\begin{bmatrix}
  2 & 7 & 1 & 2 \\
  3 & -1 & 0 & 5 \\
  1 & 7 & 3 & 3 \\
\end{bmatrix}
}

\newcommand{\QAC} {
\begin{bmatrix}
  6 & 8 & 4 \\
  -2 & 1 & 0 \\
  9 & -5 & 6 \\
\end{bmatrix}
}

%--------------------------------------------------------------------------------
\section{Perform the following matrix computation.}

\[
(6) \QAA + (-2) \QAB
\]

\sol

First of all, multiply the scalar values with the matrices
\begin{equation}
  =
  \begin{bmatrix}
    12 & -12 & 48 & 6 \\
    24 & 30 & -6 & 18 \\
    42 & -18 & 0 & 12 \\
  \end{bmatrix}
  +
  \begin{bmatrix}
    -4 & -14 & -2 & 4 \\
    -6 & 2 & 0 & -10 \\
    -2 & -14 & -6 & -6 \\
  \end{bmatrix}
\end{equation}

Then operating addition two matrices to complete

\begin{equation}
  =
  \begin{bmatrix}
    8 & -26 & 46 & 10 \\
    18 & 32 & -6 & 8 \\
    40 & -32 & -6 & 6 \\
  \end{bmatrix}
\end{equation}

\bigskip

%--------------------------------------------------------------------------------
\section{Theorem VSPM reminds you of what previous theorem? How strong is the similarity?}

\sol

It reminds of Vector Operations such as Column Vector Addition
and Column Vector Scalar Multiplication.

The domain for \(i\) is \(1 \le i \le m\),

\begin{equation}
  [u+v]_i = [u]_i + [v]_i
\end{equation}

We can perform vector scalar multiplication according to the definition of Column Vector Scalar Multiplication such that
\begin{equation}
  [\alpha u]_i = \alpha[u]_i
\end{equation}

According to the theorem, Vector Space Properties of Matrices,
we can perform operations; adding and multiplying matrices as vectors do so.

\bigskip

%--------------------------------------------------------------------------------
\section{Compute the transpose of the matrix below.}

\[
\QAC
\]

\sol

Formulating the transpose,

\begin{equation}
  \begin{bmatrix}
    6 & -2 & 9 \\
    8 & 1 & -5 \\
    4 & 0 & 6 \\
  \end{bmatrix}
\end{equation}

This is it.

\end{document}
