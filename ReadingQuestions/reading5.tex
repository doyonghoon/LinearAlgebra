\documentclass{article}
\usepackage[utf8]{inputenc}

\title{Responses for the Reading Questions 5}
\author{Yong Hoon, Do}
\date{\today}

\usepackage{amsmath}
\usepackage{amssymb}
\usepackage{systeme}

\begin{document}

\maketitle

\begin{center}
Homogeneous Systems of Equations
\end{center}

%--------------------------------------------------------------------------------
\section{What is always true of the solution set for a homogeneous system of equations?}
The solution set for a homogeneous system of equations will always be the zero vector.

%--------------------------------------------------------------------------------
\section{Suppose a homogeneous system of equations has 13 variables and 8 equations. How many solutions will it have? Why?}
It will have infinitely many solutions because

\begin{equation}
n - r > 0
\end{equation}

\begin{equation}
13 - 8 = 5
\end{equation}

\begin{equation}
5 > 0
\end{equation}

Thus, according to the definition of CSRN, there are infinitely many solutions for the system.

%--------------------------------------------------------------------------------
\section{Describe, using only words, the null space of a matrix. (So in particular, do not use any symbols.)}
The null space of matrix A is the set of all the vectors that are solutions to the homogeneous system of linear equations.

\end{document}
