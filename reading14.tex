\documentclass{article}
\usepackage[utf8]{inputenc}

\title{Responses to the Reading Questions: Matrix Multiplication}

\usepackage{bm}
\usepackage{amsmath}
\usepackage{amssymb}
\usepackage{systeme}
\usepackage{chngcntr}

\counterwithin*{equation}{section}
\counterwithin*{equation}{subsection}

\begin{document}

\maketitle

\newcommand{\sol} {
  \textbf{Solution:}
}

\newcommand{\LIVHS} {\textbf{Linearly Independent Vectors and Homogeneous Systems}}

\newcommand{\FVCS} {\textbf{Free Variables for Consistent Systems}}

\newcommand{\HSC} {\textbf{Homogeneous Systems are Consistent}}

\newcommand{\ls} {\(\mathcal{LS}(A,\textbf{0})\)}

\newcommand{\p} {$\boxed{1}$~}

\newcommand{\QAA} {
\begin{bmatrix}
  2 & 3 & -1 & 0 \\
  1 & -2 & 7 & 3 \\
  1 & 5 & 3 & 2 \\
\end{bmatrix}
}

\newcommand{\QAB} {
\begin{bmatrix}
  2 \\
  -3 \\
  0 \\
  5 \\
\end{bmatrix}
}

\newcommand{\QAC} {
\begin{bmatrix}
  6 & 8 & 4 \\
  -2 & 1 & 0 \\
  9 & -5 & 6 \\
\end{bmatrix}
}

%--------------------------------------------------------------------------------
\section{Form the matrix vector product of the following matrices}

\begin{center}
  \(\QAA\) with \(\QAB\)
\end{center}

\sol

By having an interest in realizing the size of the given matrices,

we are multiplying \(3 \times 4\) matrix with \(4 \times 1\). So we can rewrite it as

\begin{equation}
  \QAA \times \QAB
\end{equation}

Then, the resultant matrix will be \(3 \times 1\).

\begin{equation}
  \begin{bmatrix}
    (4 - 9 + 0 + 0) \\
    (2 + 6 + 0 + 15) \\
    (2 - 15 + 0 + 10) \\
  \end{bmatrix}
  =
  \begin{bmatrix}
    -5 \\
    23 \\
    -3
  \end{bmatrix}
\end{equation}

\bigskip


%--------------------------------------------------------------------------------
\section{Multiply together the two matrices below (in the order given)}

\[
\begin{bmatrix}
  2 & 3& -1 & 0 \\
  1 & -2 & 7 & 3 \\
  1 & 5 & 3 & 2
\end{bmatrix}
\hspace{1cm}
\begin{bmatrix}
  2 & 6 \\
  -3 & -4 \\
  0 & 2 \\
  3 & -1
\end{bmatrix}
\]

\sol

The first matrix is \(3 \times 4\) and the second matrix is \(4 \times 2\)
so the resultant matrix will be \(3 \times 2\) matrix.

\begin{equation}
  \begin{bmatrix}
    (4 - 9 + 0 + 0) & (12 - 12 -2 + 0) \\
    (2 + 6 + 0 + 9) & (6 + 8 + 14 - 3) \\
    (2 - 15 + 0 + 6) & (6 - 20 + 6 - 2) \\
  \end{bmatrix}
\end{equation}

This is the resultant matrix after the operations performed

\begin{equation}
  =
  \begin{bmatrix}
    -5 & -2 \\
    17 & 25 \\
    -7 & -10
  \end{bmatrix}
\end{equation}

\bigskip

%--------------------------------------------------------------------------------
\section{Rewrite the system of linear equations below as a vector equality and using a matrix-vector product. (This question does not ask for a solution to the system. But it does ask you to express the system of equations in a new form using tools from this section.)}

\[
\begin{matrix}
  2x_1 + 3x_2 - x_3 = 0 \\
  x_1 + 2x_2 + x_3 = 3 \\
  x_1 + 3x_2 + 3x_3 = 7
\end{matrix}
\]

\sol

The system of linear equations has coefficient matrix and vector of constants

\begin{equation}
  \textbf{A}=
  \begin{bmatrix}
    2 & 3& -1 \\
    1 & 2 & 1 \\
    1 & 3 & 3 \\
  \end{bmatrix}
  \hspace{1cm}
  \textbf{b}=
  \begin{bmatrix}
    0 \\
    3 \\
    7
  \end{bmatrix}
\end{equation}

so will be described compactly by the vector equation
\begin{equation}
  \textit{A}\textbf{x} = \textbf{b}
\end{equation}

Now, I am interested in finding the vector of a solution \(\mathbf{x}\) to \(\mathbf{A}\), and this is the augmented matrix

\begin{equation}
  \begin{bmatrix}
    2 & 3& -1 & 0 \\
    1 & 2 & 1 & 3 \\
    1 & 3 & 3 & 7 \\
  \end{bmatrix}
\xrightarrow[]{\text{RREF}}
\begin{bmatrix}
  \p & 0 & 0 & 1 \\
  0 & \p & 0 & 0 \\
  0 & 0 & \p & 2 \\
\end{bmatrix}
\end{equation}

so the solutions to the given system of equations is that

\begin{equation}
  \begin{bmatrix}
    x_1 \\
    x_2 \\
    x_3 \\
  \end{bmatrix}
  =
  \begin{bmatrix}
    1 \\
    0 \\
    2
  \end{bmatrix}
\end{equation}

Thus, we can write the matrix

\begin{equation}
  \begin{bmatrix}
    2 & 3& -1 \\
    1 & 2 & 1 \\
    1 & 3 & 3 \\
  \end{bmatrix}
  \begin{bmatrix}
    x_1 \\
    x_2 \\
    x_3 \\
  \end{bmatrix}
  =
  \begin{bmatrix}
    0 \\
    3 \\
    7
  \end{bmatrix}
  \xrightarrow[]{}
  \begin{bmatrix}
    2 & 3& -1 \\
    1 & 2 & 1 \\
    1 & 3 & 3 \\
  \end{bmatrix}
  \begin{bmatrix}
    1 \\
    0 \\
    2 \\
  \end{bmatrix}
  =
  \begin{bmatrix}
    0 \\
    3 \\
    7
  \end{bmatrix}
\end{equation}

\bigskip

This is the end of my homework for today. Let me go back home. Thanks for reading my homework.

\end{document}
