\documentclass{article}
\usepackage[utf8]{inputenc}

\title{Responses for the Reading Questions 7}
\author{Yong Hoon, Do}
\date{\today}

\usepackage{amsmath}
\usepackage{amssymb}
\usepackage{systeme}

\begin{document}

\maketitle

\begin{center}
Vector Operations
\end{center}

%--------------------------------------------------------------------------------
\section{Where have you seen vectors used before in other courses? How were they different?}
I have seen vectors in Calculus class and the biggest difference between those two vectors are:
\begin{itemize}
  \item The vector that I used in calculus class only includes real numbers, but the vectors that I have seen in this class includes complex numbers.
  \item The vector form is different. \(<x,y,z>\) is used as a coordinate in Euclidean \(m\)-space, but vectors taht I see in this class is used in vector space \(\mathbb{C}^m\) and writing elements vertically.
\end{itemize}

%--------------------------------------------------------------------------------
\section{In words only, when are two vectors equal?}
In Euclidean \(m\)-space, two vectors are equal if the magnitude and the direction are equal, or
two vectors are equal if every element in these two vectors at the same index are equal.

%--------------------------------------------------------------------------------
\section{Perform the following computation with vector operations}

\[
  2
    \begin{bmatrix}
      1 \\
      5 \\
      0
    \end{bmatrix}
  +
  (-3)
  \begin{bmatrix}
    7 \\
    6 \\
    5
  \end{bmatrix}
  =
  \begin{bmatrix}
    2 \\
    10 \\
    0
  \end{bmatrix}
  -
  \begin{bmatrix}
    21 \\
    18 \\
    15
  \end{bmatrix}
  =
  \begin{bmatrix}
    -19 \\
    -8 \\
    -15
  \end{bmatrix}
\]

\end{document}
