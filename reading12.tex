\documentclass{article}
\usepackage[utf8]{inputenc}

\title{Responses to the Reading Questions: Orthogonality}

\usepackage{bm}
\usepackage{amsmath}
\usepackage{amssymb}
\usepackage{systeme}
\usepackage{chngcntr}

\counterwithin*{equation}{section}
\counterwithin*{equation}{subsection}

\begin{document}

\maketitle

\newcommand{\sol} {
  \textbf{Solution:}
}

\newcommand{\LIVHS} {\textbf{Linearly Independent Vectors and Homogeneous Systems}}

\newcommand{\FVCS} {\textbf{Free Variables for Consistent Systems}}

\newcommand{\HSC} {\textbf{Homogeneous Systems are Consistent}}

\newcommand{\ls} {\(\mathcal{LS}(A,\textbf{0})\)}

\newcommand{\p} {$\boxed{1}$~}

\newcommand{\QAA} {
\begin{bmatrix}
  1 \\
  -1 \\
  2 \\
\end{bmatrix}
}

\newcommand{\QAB} {
\begin{bmatrix}
  5 \\
  3 \\
  -1 \\
\end{bmatrix}
}

\newcommand{\QAC} {
\begin{bmatrix}
  8 \\
  4 \\
  -2 \\
\end{bmatrix}
}

%--------------------------------------------------------------------------------
\noindent\textbf{Question1} is the set
\(
\begin{Bmatrix}
  \QAA, \QAB, \QAC
\end{Bmatrix}
\)

\sol

\begin{equation}
  \begin{matrix}
    <\QAA, \QAB> = 5 + (-3) + 2 = 0 \\ \\
    <\QAB, \QAC> = 40 + 12 + 2 = 54 \\ \\
    <\QAA, \QAC> = 8 - 4 -4 = 0 \\ \\
  \end{matrix}
\end{equation}

This is not an orthogonal set because one of the inner products of the set

\begin{equation}
<\QAB, \QAC> = 40 + 12 + 2 = 54 \neq 0
\end{equation}

\bigskip

\noindent\textbf{Question2} What is the distinction between an orthogonal set and an orthogonal set?

\sol

An orthonormal set is an orthogonal set but the norm of the vectors from that set is equal to 1.

\bigskip
\bigskip

\noindent\textbf{Question3} What is nice about the output of the Gram-Schmidt process?

\sol

For a linearly dependent sequence it outputs the \(O\) vector,
and for a linearly independent set of \(p\) vectors it
outputs an orthogonal set of \(p\) vectors.

\end{document}
