\documentclass{article}
\usepackage[utf8]{inputenc}

\title{Responses to the Reading Questions 9}
\author{Yong Hoon, Do}
\date{\today}

\usepackage{bm}
\usepackage{amsmath}
\usepackage{amssymb}
\usepackage{systeme}
\usepackage{chngcntr}

\counterwithin*{equation}{section}
\counterwithin*{equation}{subsection}



\begin{document}

\maketitle

\begin{center}
Spanning Sets
\end{center}

%--------------------------------------------------------------------------------
\section{Let S be the set of three vectors below.}

\[
S=
\left\{
\left|
\begin{array}{c}
  1 \\
  2 \\
  -1 \\
\end{array}
\right|
,
\left|
\begin{array}{c}
  3 \\
  -4 \\
  2 \\
\end{array}
\right|
,
\left|
\begin{array}{c}
  4 \\
  -2 \\
  1 \\
\end{array}
\right|
\right\}
\]

Let \(W=<S>\) be the span of \(S\). Is the vector
\(
\left|
\begin{array}{c}
  -1 \\ 8 \\ -4 \\
\end{array}
\right|
\)
in \(W\)? Give an explanation of the reason for your answer.

\bigskip

\textbf{Solution:}

I look for scalars \(\alpha_1\), \(\alpha_2\), \(\alpha_3\) so that
\begin{equation}
  \alpha_1 \bm{u_1} + \alpha_2 \bm{u_2} + \alpha_3 \bm{u_3} = \bm{w}
\end{equation}

By Theorem \textbf{Solutions to Linear System are Linear Combinations}, solutions to this
vector equation are solutions to the system of equations

\begin{equation}
  \begin{array}{ccc}
    \alpha_1 + 3\alpha_2 + 4\alpha_3 = -1 \\
    2\alpha_1 - 4\alpha_2 - 2\alpha_3 = 8 \\
    -1\alpha_1 + 2\alpha_2 + \alpha_3 = -4 \\
  \end{array}
\end{equation}

Building the augmented matrix for this linear system, and row-reducing, gives

\begin{equation}
 \left| \begin{array}{cccc}
  $\boxed{1}$~ & 0 & 1 & 2 \\
  0 & $\boxed{1}$~ & 1 & -1 \\
  0 & 0 & 0 & 0
  \end{array} \right|
\end{equation}

This system is consistent and has infinitely many solutions (there is a free variable in \(\alpha_3\)), but all we need is one solution vector. The solution,
\begin{equation}
  \begin{array}{c}
    \alpha_1 = 1 \\
    \alpha_2 = -2 \\
    \alpha_3 = 1 \\
  \end{array}
\end{equation}

tell us that
\begin{equation}
  (1)\bm{u_1} + (-2)\bm{u_2} + (1)\bm{u_3} = \bm{w}
\end{equation}

so we are convinced that \(\bm{w}\) really is in \(<S>\). Notice that there are an infinite number of ways to answer this question affirmatively. We could choose a different solution, this time choosing the free variable to be zero,

\begin{equation}
  (2)\bm{u_1} + (-1)\bm{u_2} + (0)\bm{u_3} = \bm{w}
\end{equation}

Verifying the arithmetic in this second solution will make it obvious that \(\bm{w}\) is in this span. And of course, we now realize that there are an infinite number of ways to realize \(\bm{w}\) as element of \(<S>\).

%--------------------------------------------------------------------------------
\section{Use S and W from the previous question. Is the vector
in \(\bm{W}\)? Give an explanation of the reason for your answer.}
The given vector is
\[
\bm{y}=
\left|
\begin{array}{c}
  6 \\ 5 \\ -1 \\
\end{array}
\right|
\]

\bigskip

\textbf{Solution:}

I look for scalars \(\alpha_1\), \(\alpha_2\), \(\alpha_3\) so that
\begin{equation}
  \alpha_1 \bm{u_1} + \alpha_2 \bm{u_2} + \alpha_3 \bm{u_3} = \bm{y}
\end{equation}

By Theorem \textbf{Solutions to Linear System are Linear Combinations}, solutions to this
vector equation are solutions to the system of equations

\begin{equation}
  \begin{array}{ccc}
    \alpha_1 + 3\alpha_2 + 4\alpha_3 = 6 \\
    2\alpha_1 - 4\alpha_2 - 2\alpha_3 = 5 \\
    -1\alpha_1 + 2\alpha_2 + \alpha_3 = -1 \\
  \end{array}
\end{equation}

Building the augmented matrix for this linear system, and row-reducing, gives

\begin{equation}
 \left| \begin{array}{cccc}
  $\boxed{1}$~ & 0 & 1 & 0 \\
  0 & $\boxed{1}$~ & 1 & 0 \\
  0 & 0 & 0 & $\boxed{1}$~
  \end{array} \right|
\end{equation}

This system is inconsistent (there is a pivot column in the last column, Theorem \textbf{RCLS}), so there are no scalars \(\alpha_1, \alpha_2, \alpha_3\) that will create a linear combination of \(\bm{u_1}, \bm{u_2}, \bm{u_3}\) that equals \(\bm{y}\). More precisely, \(\bm{y} \notin\ <S>\).

\section{For the matrix \(A\) below, find a set \(S\) so that \(<S>=\mathcal{N}(A)\), where \(\mathcal{N}(A)\) is the null space of \(A\).}
\[
A=
 \left| \begin{array}{cccc}
 1 & 3 & 1 & 9 \\
 2 & 1 & -3 & 8 \\
 1 & 1 & -1 & 5
 \end{array} \right|
\]

\textbf{Solution:}

The null space of \(A\) is the set of all solutions to the homogeneous system \(\mathcal{LS}(A,\textbf{0})\).
RREF is

\begin{equation}
 \left| \begin{array}{ccccc}
  $\boxed{1}$~ & 0 & -2 & 3 \\
  0 & $\boxed{1}$~ & 1 & 2 \\
  0 & 0 & 0 & 0
  \end{array} \right|
\end{equation}

By recognizing \(D={1,2}\), \(F={3,4}\), \(x_3\) and \(x_4\) are free variables and we can interpret each nonzero row as an expression for the deppendent variables \(x_1,x_2\) (respectively) in the free variables \(x_3,x_4\).
With this we can write the vector form of a solution vector as

\begin{equation}
  \left| \begin{array}{ccccc}
    x_1 \\
    x_2 \\
    x_3 \\
    x_4 \\
  \end{array} \right|
=
\left| \begin{array}{ccccc}
  2x_3 - 3x_4 \\
  -x_3 - 2x_4 \\
  x_3 \\
  x_4 \\
\end{array} \right|
=
x_3
\left| \begin{array}{ccccc}
  2 \\
  -1 \\
  1 \\
  0 \\
\end{array} \right|
+
x_4
\left| \begin{array}{ccccc}
  -1 \\
  -1 \\
  1 \\
  0 \\
\end{array} \right|
\end{equation}

Then in the notion of Theorem Spanning Sets for Null Spaces,

\begin{equation}
  z_1 =
  \left| \begin{array}{ccccc}
    2 \\
    -1 \\
    1 \\
    0 \\
  \end{array} \right|
  ,
  z_2 =
  \left| \begin{array}{ccccc}
    -1 \\
    -1 \\
    1 \\
    0 \\
  \end{array} \right|
\end{equation}

and
\begin{equation}
  \mathcal{N}(A) = <\{z_1,z_2\}>=
  \left< \{ \left| \begin{array}{ccccc}
    2 \\
    -1 \\
    1 \\
    0 \\
  \end{array} \right|
  ,
  \left| \begin{array}{ccccc}
    -1 \\
    -1 \\
    1 \\
    0 \\
  \end{array} \right| \} \right>
\end{equation}

\end{document}
