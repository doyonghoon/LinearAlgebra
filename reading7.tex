\documentclass{article}
\usepackage[utf8]{inputenc}

\title{Responses for the Reading Questions 7}
\author{Yong Hoon, Do}
\date{\today}

\usepackage{amsmath}
\usepackage{amssymb}
\usepackage{systeme}

\begin{document}

\maketitle

\begin{center}
Vector Operations
\end{center}

%--------------------------------------------------------------------------------
\section{Where have you seen vectors used before in other courses? How were they different?}
I have seen vectors in Calculus class and the biggest difference between those two vectors are:
\begin{itemize}
  \item The vector form is different. \(<x,y,z>\) is used as a coordinate in Euclidean \(m\)-space, but the vector taht I see in this chapter is used in vector space \(\mathbb{C}^m\)
  \item The vector \((x,y,z)\) always has an initial position in the Euclidean space.
\end{itemize}

%--------------------------------------------------------------------------------
\section{In words only, when are two vectors equal?}
Two vectors are equal if the magnitude and the direction are equal, or
two vectors are equal if every element in these two vectors in the positions of its column are equal.

%--------------------------------------------------------------------------------
\section{Perform the following computation with vector operations}

\[
  2
    \begin{bmatrix}
      1 \\
      5 \\
      0
    \end{bmatrix}
  +
  (-3)
  \begin{bmatrix}
    7 \\
    6 \\
    5
  \end{bmatrix}
  =
  \begin{bmatrix}
    2 \\
    10 \\
    0
  \end{bmatrix}
  -
  \begin{bmatrix}
    21 \\
    18 \\
    15
  \end{bmatrix}
  =
  \begin{bmatrix}
    -19 \\
    -8 \\
    -15
  \end{bmatrix}
\]

\end{document}
